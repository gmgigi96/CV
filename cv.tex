\documentclass[letterpaper]{twentysecondcv}


\profilepic{image.png}

\cvname{Gianmaria\\Del Monte}
\cvjobtitle{Software Engineer}

\cvdate{12 December 1996}
\cvaddress{Terracina (LT) - Via Marcia 23}
\cvnumberphone{+39 3386852481}
\cvsite{}
\cvgithub{gmgigi96}
\cvlinkedin{\href{https://www.linkedin.com/in/gianmaria-del-monte-81575b162/}{Gianmaria Del Monte}}
\cvmail{gia.delmonte96@gmail.com} 


\begin{document}

\aboutme{Graduated in Engineering in Computer Science with honors, I am interested in software development, focusing on analysis, design, development, testing and documentation.  Having a strong communication and leadership gained in university projects. I am a great problem solver, quickly learner, mastering new technologies. I am able to work both in a team and using own initiative.}


\hardskill{\textbf{Advanced: }Java, C, Python, Julia\\
\textbf{Intermediate: }Kotlin, Python, IA32, PostgreSQL, MongoDB, Corona SDK, Akka, OpenCL, Git, Bash, Awk, Katharà/Netkit\\
\textbf{Basis: }Flutter, LaTeX, AWS, Spark, Hive, Hadoop}

\softskill{\textbf{Problem solving: }thanks to the master degree in Engineering in CS\\
\textbf{Communication skill: }thanks to tutor experience for the Fundamentals of Computer Science course in my university\\
\textbf{Team leader: } thanks to university projects}

%----------------------------------------------------------------------------------------

\makeprofile % Print the sidebar


\section{Education}

\begin{twenty}
	\twentyitem{2019-2020}{M.Sc. in Engineering in Computer Science}{Roma Tre University}{Roma Tre University\\Thesis title: \textit{Scaling blockchains without giving up decentralization and security}\\Thesis supervisor: Prof. Maurizio Pizzonia\\Final mark: 110/110 cum laude}
	\twentyitem{2016-2018}{B.Sc. in Engineering in Computer Science}{Roma Tre University}{Thesis title: \textit{Implementation of an efficient protocol for data integrity on Cloud}\\Thesis supervisor: Prof. Maurizio Pizzonia\\Final mark: 110/110 cum laude}
	\twentyitem{2011-2015}{Scientific High school}{Terracina}{Grade: 100/100}
	%\twentyitem{<dates>}{<title>}{<location>}{<description>}
\end{twenty}


\section{Publications}

\begin{twentyshort}
	\twentyitemshort{2020}{Del Monte, Gianmaria, Diego Pennino, and Maurizio Pizzonia. "Scaling Blockchains Without Giving up Decentralization and Security." \textit{In proceeding of 3rd Workshop on Cryptocurrencies and Blockchains for Distributed Systems CryBlock (2020).} To appear.
}
	%\twentyitemshort{<dates>}{<title/description>}
\end{twentyshort}


\section{Awards}

\begin{twentyshort}
	\twentyitemshort{2019}{3rd place in local competitions of CyberChallenge 2019}
	\twentyitemshort{2018}{\textit{Luca Raso} scholarship for best students enrolled in Master's degree in Engineering in Computer Science at Roma Tre University.}
	\twentyitemshort{2016}{Scholarship for best average grade in the Engineering Department of Roma Tre University.}
\end{twentyshort}


\section{Projects}

\begin{twenty}
	\twentyitem{2018}{GelCube}{Mobile game}{A 2D platform game for both Android and iOS.}	
	\twentyitem{2018}{R3busFormazione}{Web Application}{A web application to manage a training centre, deployed on Heroku.}
	\twentyitem{2018}{Pipeline Integrity}{Cybersecurity}{Implementation of an efficient protocol for data integrity, realized with actor model, using Akka Framework and Kotlin, for my bachelor thesis.}	
	\twentyitem{2020}{Lambda Architecture}{Big Data Processing}{Implementation of a lambda architecture to make analisys on Stack Overflow dataset, deployed on AWS EMR.}	
	%\twentyitem{<dates>}{<title>}{<location>}{<description>}
\end{twenty}


\section{Relevant exams}

\textbf{Software analisys and design: } 26\\
\textbf{Software architectures: } 30 with honors\\
\textbf{Mobile computing: } 30 with honors\\
\textbf{Object oriented programming: } 30 with honors\\
\textbf{Parallel and distributed programming: } 30 with honors\\
\textbf{Databases 1/2: } 30\\
\textbf{Functional programming: } 30 with honors\\


\section{Interests}

Playing violin, classical music, etnical music, technology, Arduino

\cvprivacy

\end{document} 
